
% This LaTeX was auto-generated from MATLAB code.
% To make changes, update the MATLAB code and republish this document.

\documentclass{article}
\usepackage{graphicx}
\usepackage{color}

\sloppy
\definecolor{lightgray}{gray}{0.5}
\setlength{\parindent}{0pt}

\begin{document}

    
    \begin{verbatim}
function [J,Idata,IRegC,IRegAGlen,dIduv,IBarrierC,IBarrierAGlen]=MisfitFunction(UserVar,CtrlVar,MUA,ub,vb,ud,vd,AGlen,C,Priors,Meas)

narginchk(11,11)

us=ub+ud ; vs=vb+vd; % Surface velocities

%     J=Idata+IRegC+IRegAGlen+IBarrierC+IBarrierAGlen;
%
%
% The misfit function is:
% $$\int (u-u_d) dA$$
%
%



Area=TriAreaTotalFE(MUA.coordinates,MUA.connectivity);

switch lower(CtrlVar.MisfitFunction)

    case 'uvdiscrete'
        %
        %             residuals=[us ; vs]-[Meas.u;Meas.v];
        %             Idata=residuals'*(Cd\residuals)/2;
        %             dIduv=Cd\residuals;

        Idata=(us-Meas.us)'*(Meas.usCov\(us-Meas.us))/2+(vs-Meas.vs)'*(Meas.vsCov\(vs-Meas.vs))/2;
        dIdu=Meas.usCov\(us-Meas.us) ;
        dIdv=Meas.vsCov\(vs-Meas.vs);
        dIduv=[dIdu(:) ; dIdv(:)];

    case 'uvintegral'

        %             %fprintf(' misfit integral \n ')
        %             [n1,n2]=size(Cd);
        %             sqrtmCd=
        %             res=sqrtmCd*[us-Meas.u ; vs-Meas.v];
        %             ures=res(1:MUA.Nnodes) ;  vres=res(MUA.Nnodes+1:end) ;

        %           this sqrt only works for diagonal sparse matrices

%         isdiag was introduced in R2014a
         if isdiag(Meas.usCov)
             squs=1./sqrt(spdiags(Meas.usCov)); usres=squs.*(us-Meas.us);
         else
             error('MisfitFunction:Cov','Data covariance matrices but be diagonal')
         end

         if isdiag(Meas.vsCov)
             sqvs=1./sqrt(spdiags(Meas.vsCov)); vsres=squs.*(vs-Meas.vs);
         else
             error('MisfitFunction:Cov','Data covariance matrices must be diagonal')
         end

        M=MassMatrix2D1dof(MUA);
        Idata=full(usres'*M*usres/2+vsres'*M*vsres/2);
        dIdu=squs.*(M*usres);
        dIdv=sqvs.*(M*vsres);
        dIduv=[dIdu(:);dIdv(:)];

        if ~isreal(dIduv)
            save TestSave ; error('MisfitFunction:dIduvNoReal','dIduv is not real! Possibly a problem with covariance of data.')
        end

        % Two methods resulting in the same answer
        % method 1:


        % Method 2
        %           uNorm=sum(fgIntElementwise(connectivity,coordinates,nip,ures,ures,CtrlVar))/2;
        %           vNorm=sum(fgIntElementwise(connectivity,coordinates,nip,vres,vres,CtrlVar))/2;
        %           Idata2=(uNorm+vNorm)/Area;



        %                 Sx=StiffnessMatrixSx2D(coordinates,connectivity,nip,CtrlVar);
        %                 Sy=StiffnessMatrixSy2D(coordinates,connectivity,nip,CtrlVar);
        %
        %                 Errors=1./sqrt(spdiags(Priors.Cov,0)); Cres=(C-Priors.C)./Errors ;
        %                 Errors=1./sqrt(spdiags(Priors.CovAGlen,0)); Ares=(AGlen-Priors.AGlen)./Errors ;
        %
        %                 M=MassMatrix2D1dof(coordinates,connectivity,nip);
        %
        %                 IRegC=c0* Cres'*M*Cres/2+c1*Cres'*(Sx+Sy)*Cres/2;
        %                 IRegAGlen=a0* Ares'*M*Ares/2+a1* Ares'*(Sx+Sy)*Ares/2;
        %
        %                 IRegC=IRegC/Area;
        %                 IRegAGlen=IRegAGlen/Area;

        %Idata=FEmisfituv(ures,vres,coordinates,connectivity,nip);

        Idata=Idata/Area;
        dIduv=dIduv/Area;

    otherwise
        error(' what case? ' )
end

Idata=CtrlVar.MisfitMultiplier*Idata;


IRegC=Calc_IRegC(CtrlVar,Priors.CovC,C,Priors.C);
IRegAGlen=Calc_IRegdAGlen(CtrlVar,Priors.CovAGlen,AGlen,Priors.AGlen);
IBarrierC=Calc_IBarrierC(CtrlVar,C);
IBarrierAGlen=Calc_IBarrierAGlen(CtrlVar,AGlen);


% scalings
Idata=Idata*CtrlVar.AdjointfScale;
IRegC=IRegC*CtrlVar.AdjointfScale;
IRegAGlen=IRegAGlen*CtrlVar.AdjointfScale;
IBarrierC=IBarrierC*CtrlVar.AdjointfScale;
IBarrierAGlen=IBarrierAGlen*CtrlVar.AdjointfScale;


J=Idata+IRegC+IRegAGlen+IBarrierC+IBarrierAGlen;


%fprintf('MisfitFunction: J=%-g \t Idata=%-g \t IRegC=%-g \t IRegAGlen=%-g \t IBarrierC=%-g \t IBarrierAGlen=%-g \n',...
%    J,Idata,IRegC,IRegAGlen,IBarrierC,IBarrierAGlen)
end
\end{verbatim}



\end{document}
    
